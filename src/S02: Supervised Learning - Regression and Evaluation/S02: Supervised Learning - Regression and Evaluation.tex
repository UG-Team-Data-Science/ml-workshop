% Created 2025-06-27 Fri 10:14
% Intended LaTeX compiler: pdflatex
\documentclass[hyperref={pdfpagelabels=false},aspectratio=169]{beamer}
\usepackage[utf8]{inputenc}
\usepackage[T1]{fontenc}
\usepackage{graphicx}
\usepackage{longtable}
\usepackage{wrapfig}
\usepackage{rotating}
\usepackage[normalem]{ulem}
\usepackage{amsmath}
\usepackage{amssymb}
\usepackage{capt-of}
\usepackage{hyperref}
\usetheme{Madrid}
\usecolortheme{beaver}
\useinnertheme{rounded}
\useoutertheme{tree}
\date{\today}
\title{Machine Learning in Python}
\subtitle{Supervised Learning - Regression and Evaluation}
\author[Marocico, Tatar]{Cristian A. Marocico, A. Emin Tatar}
\institute[CIT]{Center for Information Technology\\University of Groningen}
\RequirePackage{pgfcore}
\usepackage{amsfonts}
\usepackage{amsmath}
\usepackage{amssymb}
\usepackage{blkarray}
\usepackage{eurosym}
\usepackage[english]{babel}
\usepackage{xcolor,colortbl}
\usepackage[utf8]{inputenc}
\usepackage[OT1]{fontenc}
\usepackage{multirow}
\usepackage{listings}
\usepackage{fancyvrb}
\usepackage{tikz}
\usepackage{graphicx}
\usepackage[makeroom]{cancel}
\usepackage{bbm}
\newcommand{\infoTitle}[1]{\renewcommand{\givenTitle}{#1}}
\newcommand{\givenTitle}{Info}
\newenvironment{warning}[1][Info]{%
\infoTitle{#1}
\setbeamercolor{block title}{fg=white,bg=red!100!white}%
\setbeamercolor{block body}{bg=red!10!white}
\begin{block}{\givenTitle}}{
\end{block}}
\DeclareMathOperator*{\argmax}{argmax}
\setbeamercovered{transparent}
\usecolortheme{beaver}
\RequirePackage{pgfcore}
\setbeamercovered{transparent=1}
\mode<presentation>{
\usecolortheme{beaver}
\definecolor{rugcolor}{rgb}{0.8,0,0}
\definecolor{darkblue}{rgb}{0.13,0.29,0.53}
\definecolor{darkgreen}{rgb}{0.0,0.43,0.0}
\definecolor{darkyellow}{rgb}{0.0,0.43,0.43}
\definecolor{codegreen}{rgb}{0,0.6,0}
\definecolor{codegray}{rgb}{0.5,0.5,0.5}
\definecolor{codepurple}{rgb}{0.58,0,0.82}
\definecolor{backcolour}{rgb}{0.95,0.95,0.92}
\definecolor{Gray}{gray}{0.85}
\beamertemplatenavigationsymbolsempty
\setbeamercolor{item}{fg=rugcolor!80!black}
\setbeamercolor{title}{fg=rugcolor!80!black}
\setbeamercolor{frametitle}{fg=rugcolor!80!black, bg=black!10!white}
% Colors for 'definition' environment
\setbeamercolor*{block title}{fg=white, bg=darkblue}
\setbeamercolor*{block body}{bg=normal text.bg!85!darkblue}
% Color for the 'question' environment
\setbeamercolor*{block title question}{fg=white, bg=darkyellow}
\setbeamercolor*{block body question}{bg=normal text.bg!85!darkyellow}
\setbeamercolor*{palette tertiary}{bg=rugcolor,fg=white}
%HEADER WITH HIGHLIGHTED SECTION NAMES (optional)
\useheadtemplate{%
\vbox{%
%			\vskip1.2pt
%			\pgfuseimage{logo}
%			\vskip1.2pt
\tinycolouredline{rugcolor}{
\color{white}{
% comment the following line if you don't want the section names lines
%					to appear on top
\insertsectionnavigationhorizontal{\paperwidth}{}{\hskip0pt
plus1filll}
%\pgfuseimage{logored}
}
}
%    \tinycolouredline{rugcolor}
{\color{white}{
%% \insertsectionnavigationhorizontal{\paperwidth}{}{
%                \hskip0pt \hfill}
}}
}
}
%FOOTER WITH AUTHOR NAME(S), PAPER TITLE (ABBREVIATED IF SPECIFIED BY \title),
% AND PAGE COUNTER (optional)
%	\usefoottemplate{%
%		\vbox{%
%			\tinycolouredline{rugcolor}{
%				\color{white}{
%					{\insertshortauthor} \hfill \insertshortsubtitle \hfill
%					%\insertdate \hfill%
%					\textsc{\insertframenumber/\inserttotalframenumber}
%         		}
%         	}
%		}
%	}
}
\newtheorem*{props}{Properties}
\newtheorem*{prop}{Property}
\newtheorem*{notation}{Notation}
\newtheorem*{terminology}{Terminology}
\newcolumntype{a}{>{\columncolor{Gray}}c}
\DeclareMathOperator*{\argmin}{argmin}
\newcommand{\Var}{\mathbb{V}\mathrm{ar}}
\newcommand{\Corr}{\mathbb{C}\mathrm{orr}}
\newcommand{\Cov}{\mathbb{C}\mathrm{ov}}
\newcommand{\Expt}{\mathbb{E}}
\newcommand{\NorDist}{\mathcal{N}}
\newcommand{\ExpDist}{\mathcal{E}\mathrm{xp}}
\newcommand{\GammaDist}{\mathcal{G}\mathrm{amma}}
\newcommand{\BetaDist}{\mathcal{B}\mathrm{eta}}
\newcounter{listCounter}
\newenvironment{question}{%
\setbeamercolor{block title}{bg=orange!70!white,fg=white}
\setbeamercolor{block body}{bg=yellow!10!white}
\begin{block}{Question}
}{%
\end{block}
}
\lstdefinestyle{mystyle}{
language=R,
backgroundcolor=\color{backcolour},
commentstyle=\color{codegreen},
keywordstyle=\color{darkblue}\bfseries,
numberstyle=\tiny\color{codegray},
stringstyle=\color{codepurple},
basicstyle=\ttfamily,
breakatwhitespace=true,
breaklines=true,
%	captionpos=none,
columns=fixed,
keepspaces=true,
numbers=none,
numbersep=5pt,
showspaces=false,
showstringspaces=false,
showtabs=false,
tabsize=2,
basewidth=1.5em,
escapeinside={<@}{@>}
}
\setbeamertemplate{caption}{\raggedright\insertcaption\par}
\setbeamertemplate{blocks}[rounded][shadow=true]
\renewenvironment{definition}[1][Definition]{%
\setbeamercolor{block title}{fg=white,bg=darkblue}
\setbeamercolor{block body}{fg=black,bg=normal text.bg!85!darkblue}
\begin{block}{#1\hfill \footnotesize{Definition}}
\vspace*{-5pt}
}{%
\end{block}
}
\renewenvironment{example}[1][Example]{%
% Color for the 'example' environment
\setbeamercolor*{block title}{fg=white, bg=darkgreen}
\setbeamercolor*{block body}{bg=normal text.bg!85!darkgreen}
\begin{block}{#1\hfill \footnotesize{Example}}
\vspace*{-5pt}
}{%
\end{block}
}
\renewenvironment{theorem}[1][Theorem]{%
\setbeamercolor*{block title}{fg=white, bg=darkyellow}
\setbeamercolor*{block body}{bg=normal text.bg!85!darkyellow}
\begin{block}{#1\hfill \footnotesize{Theorem}}
\vspace*{-5pt}
}{%
\end{block}
}
\date[Jul 2\textsuperscript{nd} 2025]{Wednesday, July 2\textsuperscript{nd} 2025}
\usepackage{mathtools}
\newcommand{\intsum}{\mathop{\mathrlap{\raisebox{0.1ex}{\hspace{0.2em}$\textstyle\sum$}}\int}\limits}
\setbeamercovered{transparent=0}
\usepackage[timeinterval=60]{tdclock}
\hypersetup{
 pdfauthor={},
 pdftitle={Machine Learning in Python},
 pdfkeywords={},
 pdfsubject={},
 pdfcreator={Emacs 30.1 (Org mode 9.7.11)}, 
 pdflang={English}}
\begin{document}

\maketitle
\begin{frame}{Outline}
\setcounter{tocdepth}{1}
\tableofcontents
\end{frame}

\section{Introduction to Regression}
\label{sec:org0b72ec1}
\begin{frame}[label={sec:org6d6ed3f}]{Introduction to Regression}
\begin{definition}[Regression]\label{sec:org970c3a8}
\pause
\alert{Regression} is a statistical method used to model the relationship between a dependent variable and one or more independent variables.
\end{definition}
\end{frame}
\section{Simple Linear Regression}
\label{sec:org2ad28d3}
\begin{frame}[label={sec:orgbdf2883}]{Simple Linear Regression}
\begin{definition}[Simple Linear Regression]\label{sec:orgbad44f8}
\pause
\alert{Simple Linear Regression} is a method to model the relationship between two variables by fitting a linear equation to observed data.
\pause
Mathematically:
$$y = \beta_0 + \beta_1 x + \epsilon$$
where:
\begin{itemize}
\item \(y\) is the dependent variable (response).
\item \(x\) is the independent variable (predictor).
\item \(\beta_0\) is the y-intercept (constant term).
\item \(\beta_1\) is the slope of the line (coefficient).
\item \(\epsilon\) is the error term (residuals).
\end{itemize}
\end{definition}
\end{frame}
\begin{frame}[label={sec:orge9eb242}]{Simple Linear Regression}
A Simple Linear Regression Machine Learning model will learn the coefficients \(\beta_0\) and \(\beta_1\) from the training data to minimize the difference between the predicted values and the actual values.
\end{frame}
\begin{frame}[label={sec:orgb40475f}]{Assumptions of Simple Linear Regression}
\begin{itemize}[<+->]
\item Linearity: The relationship between the independent and dependent variable is linear.
\item Independence: Observations are independent of each other.
\item Homoscedasticity: Constant variance of the error terms.
\item Normality: The residuals (errors) of the model are normally distributed.
\end{itemize}
\end{frame}
\section{Evaluation Metrics for Regression}
\label{sec:org7df6b85}
\begin{frame}[label={sec:org78ab07a}]{Evaluation Metrics for Regression}
\begin{definition}[Common Metrics for Regression]\label{sec:org80fe02b}
\pause
Common metrics to evaluate regression models include:
\begin{itemize}
\item \alert{Mean Absolute Error (MAE)}
\item \alert{Mean Squared Error (MSE)}
\item \alert{Root Mean Squared Error (RMSE)}
\item \alert{R-squared (\(R^2\))}
\item \alert{Adjusted R-squared}
\end{itemize}
\end{definition}
\end{frame}
\begin{frame}[label={sec:orgb3bc797}]{Mean Absolute Error (MAE)}
\begin{definition}[Mean Absolute Error (MAE)]\label{sec:org14a3c74}
\pause
\alert{Mean Absolute Error (MAE)} is the average of the absolute differences between the predicted and the actual values:
$$\text{MAE} = \frac{1}{n} \sum_{i=1}^{n} |y_i - \hat{y}_i|$$
\pause
where:
\begin{itemize}
\item \(n\) is the number of observations.
\item \(y_i\) is the actual value.
\item \(\hat{y}_i\) is the predicted value.
\end{itemize}
\pause
MAE is a linear score, which can be used when all errors are equally important; it is also less sensitive to outliers compared to MSE.
\end{definition}
\end{frame}
\begin{frame}[label={sec:org68791bd}]{Mean Squared Error (MSE)}
\begin{definition}[Mean Squared Error (MSE)]\label{sec:org0ac2673}
\pause
\alert{Mean Squared Error (MSE)} is the average of the squared differences between the predicted and the actual values:
$$\text{MSE} = \frac{1}{n} \sum_{i=1}^{n} (y_i - \hat{y}_i)^2$$
\pause
where:
\begin{itemize}
\item \(n\) is the number of observations.
\item \(y_i\) is the actual value.
\item \(\hat{y}_i\) is the predicted value.
\end{itemize}
\pause
MSE is more sensitive to outliers than MAE because it squares the errors, which can disproportionately affect the metric if there are large errors; however, it is useful when larger errors are more significant.
\end{definition}
\end{frame}
\begin{frame}[label={sec:orgbcdf310}]{Root Mean Squared Error (RMSE)}
\begin{definition}[Root Mean Squared Error (RMSE)]\label{sec:orgf25b7a1}
\pause
\alert{Root Mean Squared Error (RMSE)} is the square root of the average of the squared differences between the predicted and the actual values:
$$\text{RMSE} = \sqrt{\text{MSE}} = \sqrt{\frac{1}{n} \sum_{i=1}^{n} (y_i - \hat{y}_i)^2}$$
\pause
where:
\begin{itemize}
\item \(n\) is the number of observations.
\item \(y_i\) is the actual value.
\item \(\hat{y}_i\) is the predicted value.
\end{itemize}
\pause
RMSE is in the same units as the dependent variable, making it interpretable; it is also sensitive to outliers, similar to MSE.
\end{definition}
\end{frame}
\begin{frame}[label={sec:org505fc78}]{R-squared (\(R^2\))}
\begin{definition}[R-squared (\(R^2\))]\label{sec:org38c5b60}
\pause
\alert{R-squared (\(R^2\))} is a statistical measure that represents the proportion of the variance for a dependent variable that's explained by an independent variable or variables in a regression model:
$$R^2 = 1 - \frac{\text{SS}_{\text{res}}}{\text{SS}_{\text{tot}}}$$
where:
\begin{itemize}
\item \(\text{SS}_{\text{res}}\) is the sum of squares of residuals (errors).
\item \(\text{SS}_{\text{tot}}\) is the total sum of squares (variance of the dependent variable).
\end{itemize}
\pause
\(R^2\) values range from 0 to 1, where:
\begin{itemize}
\item 0 indicates that the model does not explain any of the variability of the response data around its mean.
\item 1 indicates that the model explains all the variability of the response data around its mean.
\end{itemize}
\end{definition}
\end{frame}
\begin{frame}[label={sec:orge1342a0}]{Adjusted R-squared}
\begin{definition}[Adjusted R-squared]\label{sec:orge0526cd}
\pause
\alert{Adjusted R-squared} adjusts the \(R^2\) value for the number of predictors in the model, providing a more accurate measure when comparing models with different numbers of predictors:
$$\text{Adjusted } R^2 = 1 - \left(1 - R^2\right) \frac{n - 1}{n - p - 1}$$
where:
\begin{itemize}
\item \(n\) is the number of observations.
\item \(p\) is the number of predictors in the model.
\end{itemize}
\pause
Adjusted \(R^2\) can be negative, which indicates that the model is worse than a horizontal line (mean of the dependent variable); it is useful for comparing models with different numbers of predictors.
\end{definition}
\end{frame}
\end{document}
