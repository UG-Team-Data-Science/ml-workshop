% Created 2025-07-03 Thu 20:15
% Intended LaTeX compiler: pdflatex
\documentclass[hyperref={pdfpagelabels=false},aspectratio=169]{beamer}
\usepackage[utf8]{inputenc}
\usepackage[T1]{fontenc}
\usepackage{graphicx}
\usepackage{longtable}
\usepackage{wrapfig}
\usepackage{rotating}
\usepackage[normalem]{ulem}
\usepackage{amsmath}
\usepackage{amssymb}
\usepackage{capt-of}
\usepackage{hyperref}
\usetheme{Madrid}
\usecolortheme{beaver}
\useinnertheme{rounded}
\useoutertheme{tree}
\date{\today}
\title{Machine Learning in Python}
\subtitle{Ensemble Methods}
\author[Marocico, Tatar]{Cristian A. Marocico, A. Emin Tatar}
\institute[CIT]{Center for Information Technology\\University of Groningen}
\RequirePackage{pgfcore}
\usepackage{amsfonts}
\usepackage{amsmath}
\usepackage{amssymb}
\usepackage{blkarray}
\usepackage{eurosym}
\usepackage[english]{babel}
\usepackage{xcolor,colortbl}
\usepackage[utf8]{inputenc}
\usepackage[OT1]{fontenc}
\usepackage{multirow}
\usepackage{listings}
\usepackage{fancyvrb}
\usepackage{tikz}
\usepackage{graphicx}
\usepackage[makeroom]{cancel}
\usepackage{bbm}
\newcommand{\infoTitle}[1]{\renewcommand{\givenTitle}{#1}}
\newcommand{\givenTitle}{Info}
\newenvironment{warning}[1][Info]{%
\infoTitle{#1}
\setbeamercolor{block title}{fg=white,bg=red!100!white}%
\setbeamercolor{block body}{bg=red!10!white}
\begin{block}{\givenTitle}}{
\end{block}}
\DeclareMathOperator*{\argmax}{argmax}
\setbeamercovered{transparent}
\usecolortheme{beaver}
\RequirePackage{pgfcore}
\setbeamercovered{transparent=1}
\mode<presentation>{
\usecolortheme{beaver}
\definecolor{rugcolor}{rgb}{0.8,0,0}
\definecolor{darkblue}{rgb}{0.13,0.29,0.53}
\definecolor{darkgreen}{rgb}{0.0,0.43,0.0}
\definecolor{darkyellow}{rgb}{0.0,0.43,0.43}
\definecolor{codegreen}{rgb}{0,0.6,0}
\definecolor{codegray}{rgb}{0.5,0.5,0.5}
\definecolor{codepurple}{rgb}{0.58,0,0.82}
\definecolor{backcolour}{rgb}{0.95,0.95,0.92}
\definecolor{Gray}{gray}{0.85}
\beamertemplatenavigationsymbolsempty
\setbeamercolor{item}{fg=rugcolor!80!black}
\setbeamercolor{title}{fg=rugcolor!80!black}
\setbeamercolor{frametitle}{fg=rugcolor!80!black, bg=black!10!white}
% Colors for 'definition' environment
\setbeamercolor*{block title}{fg=white, bg=darkblue}
\setbeamercolor*{block body}{bg=normal text.bg!85!darkblue}
% Color for the 'question' environment
\setbeamercolor*{block title question}{fg=white, bg=darkyellow}
\setbeamercolor*{block body question}{bg=normal text.bg!85!darkyellow}
\setbeamercolor*{palette tertiary}{bg=rugcolor,fg=white}
%HEADER WITH HIGHLIGHTED SECTION NAMES (optional)
\useheadtemplate{%
\vbox{%
%			\vskip1.2pt
%			\pgfuseimage{logo}
%			\vskip1.2pt
\tinycolouredline{rugcolor}{
\color{white}{
% comment the following line if you don't want the section names lines
%					to appear on top
\insertsectionnavigationhorizontal{\paperwidth}{}{\hskip0pt
plus1filll}
%\pgfuseimage{logored}
}
}
%    \tinycolouredline{rugcolor}
{\color{white}{
%% \insertsectionnavigationhorizontal{\paperwidth}{}{
%                \hskip0pt \hfill}
}}
}
}
%FOOTER WITH AUTHOR NAME(S), PAPER TITLE (ABBREVIATED IF SPECIFIED BY \title),
% AND PAGE COUNTER (optional)
%	\usefoottemplate{%
%		\vbox{%
%			\tinycolouredline{rugcolor}{
%				\color{white}{
%					{\insertshortauthor} \hfill \insertshortsubtitle \hfill
%					%\insertdate \hfill%
%					\textsc{\insertframenumber/\inserttotalframenumber}
%         		}
%         	}
%		}
%	}
}
\newtheorem*{props}{Properties}
\newtheorem*{prop}{Property}
\newtheorem*{notation}{Notation}
\newtheorem*{terminology}{Terminology}
\newcolumntype{a}{>{\columncolor{Gray}}c}
\DeclareMathOperator*{\argmin}{argmin}
\newcommand{\Var}{\mathbb{V}\mathrm{ar}}
\newcommand{\Corr}{\mathbb{C}\mathrm{orr}}
\newcommand{\Cov}{\mathbb{C}\mathrm{ov}}
\newcommand{\Expt}{\mathbb{E}}
\newcommand{\NorDist}{\mathcal{N}}
\newcommand{\ExpDist}{\mathcal{E}\mathrm{xp}}
\newcommand{\GammaDist}{\mathcal{G}\mathrm{amma}}
\newcommand{\BetaDist}{\mathcal{B}\mathrm{eta}}
\newcounter{listCounter}
\newenvironment{question}{%
\setbeamercolor{block title}{bg=orange!70!white,fg=white}
\setbeamercolor{block body}{bg=yellow!10!white}
\begin{block}{Question}
}{%
\end{block}
}
\lstdefinestyle{mystyle}{
language=R,
backgroundcolor=\color{backcolour},
commentstyle=\color{codegreen},
keywordstyle=\color{darkblue}\bfseries,
numberstyle=\tiny\color{codegray},
stringstyle=\color{codepurple},
basicstyle=\ttfamily,
breakatwhitespace=true,
breaklines=true,
%	captionpos=none,
columns=fixed,
keepspaces=true,
numbers=none,
numbersep=5pt,
showspaces=false,
showstringspaces=false,
showtabs=false,
tabsize=2,
basewidth=1.5em,
escapeinside={<@}{@>}
}
\setbeamertemplate{caption}{\raggedright\insertcaption\par}
\setbeamertemplate{blocks}[rounded][shadow=true]
\renewenvironment{definition}[1][Definition]{%
\setbeamercolor{block title}{fg=white,bg=darkblue}
\setbeamercolor{block body}{fg=black,bg=normal text.bg!85!darkblue}
\begin{block}{#1\hfill \footnotesize{Definition}}
\vspace*{-5pt}
}{%
\end{block}
}
\renewenvironment{example}[1][Example]{%
% Color for the 'example' environment
\setbeamercolor*{block title}{fg=white, bg=darkgreen}
\setbeamercolor*{block body}{bg=normal text.bg!85!darkgreen}
\begin{block}{#1\hfill \footnotesize{Example}}
\vspace*{-5pt}
}{%
\end{block}
}
\renewenvironment{theorem}[1][Theorem]{%
\setbeamercolor*{block title}{fg=white, bg=darkyellow}
\setbeamercolor*{block body}{bg=normal text.bg!85!darkyellow}
\begin{block}{#1\hfill \footnotesize{Theorem}}
\vspace*{-5pt}
}{%
\end{block}
}
\date[Jul 7\textsuperscript{th} 2025]{Monday, July 7\textsuperscript{th} 2025}
\usepackage{mathtools}
\newcommand{\intsum}{\mathop{\mathrlap{\raisebox{0.1ex}{\hspace{0.2em}$\textstyle\sum$}}\int}\limits}
\setbeamercovered{transparent=0}
\usepackage[timeinterval=60]{tdclock}
\hypersetup{
 pdfauthor={},
 pdftitle={Machine Learning in Python},
 pdfkeywords={},
 pdfsubject={},
 pdfcreator={Emacs 30.1 (Org mode 9.7.11)}, 
 pdflang={English}}
\begin{document}

\maketitle
\begin{frame}{Outline}
\setcounter{tocdepth}{1}
\tableofcontents
\end{frame}

\section{Introduction to Ensemble Methods}
\label{sec:org4f0fa4f}
\begin{frame}[label={sec:org86a52e0}]{Introduction to Ensemble Methods}
\begin{definition}[What are Ensemble Methods?]\label{sec:org0b5ec31}
\pause
\alert{Ensemble methods} combine multiple models to improve predictive performance. They leverage the strengths of individual models while mitigating their weaknesses.
\end{definition}
\end{frame}
\begin{frame}[label={sec:org4cdbf2f}]{Introduction to Ensemble Methods}
\begin{example}[Why Use Ensemble Methods?]\label{sec:org8f1545c}
\begin{itemize}[<+->]
\item \alert{Improved Accuracy}: Ensemble methods often outperform individual models by reducing variance and bias.
\item \alert{Robustness}: Less sensitive to noise and outliers, leading to more stable predictions.
\item \alert{Reduced Overfitting}: Reduce overfitting by averaging predictions from multiple models.
\item \alert{Flexibility}: Ensemble methods can be applied to various base models, including decision trees, linear models, and neural networks.
\item \alert{Interpretability}: Some ensemble methods, like Random Forests, provide insights into feature importance.
\item \alert{Scalability}: Many ensemble methods can be parallelized, useful for large datasets.
\item \alert{Combining Different Algorithms}: Ensembles can combine predictions from different types of models, enhancing overall performance.
\end{itemize}
\end{example}
\end{frame}
\begin{frame}[label={sec:org8538f28}]{Types of Ensemble Methods}
\begin{example}[Ensamble Methods]\label{sec:org446db78}
\pause
Ensemble methods can be broadly categorized into several types, each with its own approach to combining models:
\begin{itemize}[<+->]
\item \alert{Bagging (Bootstrap Aggregating)}: Combines predictions from multiple models trained on different subsets of the training data. The most common example is Random Forests.
\item \alert{Boosting}: Sequentially builds models, where each new model focuses on correcting the errors made by the previous models. Examples include AdaBoost, Gradient Boosting, and XGBoost.
\item \alert{Stacking}: Combines multiple models by training a meta-model on their predictions. The base models can be of different types, and the meta-model learns how to best combine their outputs.
\item Other methods include \alert{Pasting}, \alert{Bayesian Model Averaging}, \alert{Blending}, and \alert{Voting}.
\end{itemize}
\end{example}
\end{frame}
\section{Bagging}
\label{sec:org77e6e44}
\begin{frame}[label={sec:orgb219e80}]{Bootstrap Aggregating (Bagging) Overview}
\begin{definition}[Bootstrap Aggregating (Bagging)]\label{sec:org3d5a40b}
\pause
\alert{Bagging}, or \alert{Bootstrap Aggregating}, is an ensemble method that combins the predictions of multiple models trained on different subsets of the training data. The key steps in bagging are:
\pause
\begin{itemize}[<+->]
\item \alert{Bootstrap Sampling}: Create multiple bootstrap samples from the training data. Each sample is created by randomly selecting instances with replacement, meaning some instances may appear multiple times while others may not appear at all.
\item \alert{Model Training}: Train a separate model on each bootstrap sample. These models can be of the same type (e.g., decision trees) or different types.
\item \alert{Aggregation}: Combine the predictions of the individual models. For regression tasks, this is typically done by averaging the predictions, while for classification tasks, a majority vote is used to determine the final class label.
\end{itemize}
\end{definition}
\end{frame}
\begin{frame}[label={sec:org2795890}]{Bagging and Random Forests}
\begin{example}[Bagging and Random Forests]\label{sec:org3a7d282}
\pause
\alert{Random Forests} is a popular ensemble method that builds multiple decision trees using bagging. It introduces additional randomness by selecting a random subset of features for each tree, which helps to decorrelate the trees and improve overall model performance.
\end{example}
\end{frame}
\begin{frame}[label={sec:org7824010}]{Feature Importance in Random Forests}
\begin{block}{Feature Importance in Random Forests}
\pause
\alert{Random Forests} can also provide insights into feature importance, which helps in understanding which features contribute most to the predictions. This is done by measuring the decrease in model performance when a feature is permuted or removed.
\end{block}
\end{frame}
\section{Boosting}
\label{sec:org354da6d}
\begin{frame}[label={sec:orgad20352}]{Boosting Overview}
\begin{definition}[Boosting Overview]\label{sec:org7285c14}
\pause
\alert{Boosting} is an ensemble method that builds models sequentially, where each new model focuses on correcting the errors made by the previous models. The key steps in boosting are:
\pause
\begin{itemize}[<+->]
\item \alert{Sequential Model Training}: Train a series of models, where each model is trained on the residuals (errors) of the previous model. This allows each new model to learn from the mistakes of its predecessor.
\item \alert{Weighted Predictions}: Each model's predictions are weighted based on its performance. Models that perform well have a higher weight in the final prediction.
\item \alert{Final Prediction}: Combine the predictions of all models, typically by summing their weighted predictions. For classification tasks, a threshold is applied to determine the final class label.
\end{itemize}
\end{definition}
\end{frame}
\begin{frame}[label={sec:org0218e49}]{Boosting Algorithms}
\begin{example}[Boosting Algorithms]\label{sec:orgac5a91e}
\pause
Some popular boosting algorithms include:
\begin{itemize}[<+->]
\item \alert{AdaBoost}: Assigns weights to instances based on their classification errors, focusing more on misclassified instances in subsequent iterations.
\item \alert{Gradient Boosting}: Builds models that minimize a loss function by fitting to the residuals of the previous model. It can handle various loss functions, making it versatile for regression and classification tasks.
\item \alert{XGBoost}: An optimized implementation of gradient boosting that includes regularization, handling missing values, and parallel processing for faster training.
\end{itemize}
\end{example}
\end{frame}
\section{Ensembles of Ensemble Methods}
\label{sec:orge8cb852}
\begin{frame}[label={sec:orgf4d1493}]{Ensembles of Ensemble Methods}
\begin{example}[Ensembles of Ensemble Methods]\label{sec:org13f879a}
\pause
Ensembles of ensemble methods combine multiple ensemble techniques to further enhance predictive performance. This approach can lead to even more robust models by leveraging the strengths of different ensemble methods.
\end{example}
\end{frame}
\begin{frame}[label={sec:orgfd5a679}]{Ensembles of Ensemble Methods}
\begin{block}{Ensembles of Ensemble Methods}
\pause
Ensembles of ensemble methods can be created by combining different types of ensembles, such as bagging and boosting, or by stacking multiple ensemble models. This approach allows for greater diversity in the predictions and can lead to improved generalization performance.
\end{block}
\end{frame}
\end{document}
